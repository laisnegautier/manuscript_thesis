
\chapter*{Notations}
\addcontentsline{toc}{chapter}{Notations}
\label{chapter:notations}

Throughout this thesis, numerous mathematical objects (vectors, matrices, lines, sets, etc) are manipulated and sometimes abbreviated into other notations.
This chapter helps understand how such objects are notated and how to recognize from which category they belong to.

\begin{itemize}
    \item {\textbf{Common sets of numbers}: in uppercase and blackboard font. Examples: natural numbers $\mathbb{N}$, real numbers $\mathbb{R}$;}
    \item {\textbf{Bounded convex sets}: in uppercase, italic and caligraphic. Examples: cube $\mathcal{C}$, sphere $\mathcal{S}$,
    polytope $\mathcal{P}$, zonotope $\mathcal{Z}$, orthotope $\mathcal{O}$;}
    \item{\textbf{Affine and vector spaces}: in uppercase, italic, block letters. Examples: euclidean space $E$, line $L$, plane $P$, hyperplane $H$;}
    \item {\textbf{Coefficients}: in lowercase, italic and greek. Examples: $\lambda,\, \alpha,\, \mu \in\mathbb{R}$;}
    \item {\textbf{Points, integers and real numbers}: in lowercase, italic and latin. Examples: the points $p,\, q,\, r \in E$, the integers $n,\,m\in\mathbb{N}$ and the real $x\in \mathbb{R}$;}
    \item {\textbf{Vectors}: in lowercase, bold and latin. Vectors can be indexed by non-bold characters when needed. Examples: $\mathbf{x},\, \mathbf{y},\, \mathbf{q} \in \mathbb{R}^n$ and $\mathbf{x} = (x_1,\, \dots,\, x_n)$;}
    \item {\textbf{Matrices}: in uppercase, italic and block letters. Example: $M \in \mathbb{R}^{n\times m}$;}
    \item {\textbf{Affine maps}: in uppercase, italic and block letters. They usually are described using their matrice equivalent, hence we omit the needed parentheses to precise the mapped object.
    Examples: a linear transformation $N$ on a cube $\mathcal{C}$ is $N\mathcal{C}$, a dilation $D$ on a sphere $\mathcal{S}$ is $D\mathcal{S}$. 
    An affine map can contain a translational part, which is voluntarily ommitted in the notation;}
    \item {\textbf{Functions}: in lowercase, italic and in latin or greek, with parentheses. Examples: the function $f\colon \mathbb{R}\rightarrow \mathbb{R}$ such that $f(x) = x^3$.}
\end{itemize}

\textbf{Warnings on affine maps:} Specific invertible affine maps are rotations and translations, but both are considered as rotations in the notation. 
A rototranslation $R$ applied on a point $p$, a vector $\mathbf{x}$, a plane $P$ or a cube $\mathcal{C}$ is 
respectively notated $Rp$, $R\mathbf{x}$, $RP$ and $R\mathcal{C}$. 
The letter $R$ references \textit{rotor} and not \textit{rotation}. Rotors are a generalization of rotations to any dimension, and includes translations to some extent. 
The rotor construction will not be discussed in this work. For more information about \textit{geometric algebra}, rotors and general euclidean transformations, the reader is 
invited to have a look at \cite{dorstGeometricAlgebraComputer2007}.

While conciseness and compactness are favored in most mathematical expressions presented in this work, there are cases where
explicitness should be prevalent. As such, notations may differ but it will be clearly mentionned. 
