\chapter*{Conclusion}
\markboth{Conclusion}{Conclusion}
\addcontentsline{toc}{chapter}{Conclusion}
\label{chapter:conclusion}
This thesis explores the potential of physical Human-Robot Interaction (pHRI) in the modern industry. In collaborative settings involving humans and robots, robots are increasingly capable of adapting to dynamic scenarios and offer a unique opportunity to assist human operators by accommodating their physical needs, preferences, and skill levels. However, this flexibility in collaborative workspaces raises safety concerns. To ensure that robotic assistance is not a burden on the human operator, the robot must possess a clear and precise understanding of human capabilities. In particular, knowledge of an operator's physical abilities, such as their force capacities, is crucial.

This thesis focuses on a robotic-based formulation of these force capacities at the hand, employing a set-theoretic framework. Accurately capturing human force capabilities requires experimental measurement. For a fixed upper-limb posture, this involves repeated exertion of maximal isometric forces in various directions. This leads to a first challenge addressed in this thesis:
\begin{mdframed}
    \begin{center}
        \textbf{Challenge 1:} \\
        How can the measurement of maximal isometric forces be performed efficiently within a reasonably long experiment?
    \end{center}
\end{mdframed}

Collaborative tasks demand varying levels of human physical involvement, necessitating adaptable robot assistance. Given that an individual's force capabilities are influenced by posture, physiology, and anthropometry, the robot should account for these factors. To address this inter- and intra-individual variability, this thesis employs a numerical representation of the human, known as a musculoskeletal model. However, the computational cost of such models increases with their complexity and level of detail. Thus, this thesis addresses this second challenge:
\begin{mdframed}
    \begin{center}
        \textbf{Challenge 2:} \\
        How detailed and personalized must a musculoskeletal model of the human upper-limb be to accurately represent the exertable maximal isometric forces at the hand of an individual?
    \end{center}
\end{mdframed}   

Musculoskeletal modeling offers valuable insights into human biomechanics. A deeper understanding of exerted forces can enhance the design and control of collaborative robots, enabling safer and more effective human-robot interaction. This thesis explores the potential of a set-theoretic formulation of exertable forces to understand how muscle interactions contribute to maximal force production. This leads to the third challenge addressed in this thesis:
\begin{mdframed}
    \begin{center}
        \textbf{Challenge 3:} \\
        How can a set-theoretic approach to maximal isometric forces be used to quantitatively characterize muscle tension interactions?
    \end{center}
\end{mdframed}


\subsection*{Thesis contributions}
\addcontentsline{toc}{section}{Thesis contributions}
In Chapter \ref{chapter:1}, we described how maximal isometric forces could be formulated within a set-theoretic approach using musculoskeletal models. These sets, termed \emph{force feasible sets}, are studied in the literature within both robotic and biomechanical contexts. We first described the experimental protocol for collecting maximal isometric forces at the hand and reviewed the factors that influenced the quality of an exerted maximal force. By comparing experimental measurements from robotics and biomechanics, we highlighted a discrepancy between musculoskeletal-based force feasible sets and experimental data. This discrepancy arise from differing biomechanical assumptions regarding muscle interactions, leading to distinct characterizations of force feasible sets, notably as (convex) polytopes or ellipsoids.

Chapter \ref{chapter:2} focused on improving the computational aspects of force polytopes to better understand the combinatorial geometric processes involved in their formulation. We presented a new, efficient algorithm for computing the vertices of a zonotope — the projection of a hyperrectangle — which could represent feasible torques. The efficiency of this algorithm was theoretically proven using algorithmic complexity analysis, positioning our approach within the context of recent advances in the field. This chapter elucidated the computational challenges associated with describing torque feasible sets, which are inherently linked to polytopic representations of force feasible sets.

While force feasible sets modeled as polytopes assume independent muscle tensions, Chapter \ref{chapter:3} explored alternative representations of muscle tension interactions. Adopting a more theoretical perspective, this chapter integrated mathematical results and their biomechanical implications into the framework of \emph{in silico} force feasible sets. We argued that a large number of muscles in a musculoskeletal model permitted the representation of a broad class of force feasible sets as ellipsoids, with a scaling factor indicative of the level of muscle tension interaction. While these shape-related results partially address Challenges 1 and 2, we further investigated this interaction by explicitly computing the scaling factor, termed the \emph{projection constant}. This computation enables a numerical transition from a polytopic to an ellipsoidal representation, potentially mitigating the computational complexity associated with polytopes.

Furthermore, this chapter presented two additional results derived from this new class of representations. The first result provides a deeper understanding of how geometric assumptions about force feasible sets were reflected in muscle tension interaction models. It involves an explicit computation of force feasible sets in the muscle tension space, revealing how muscle tensions were linearly constrained when producing a maximal force. This analysis also demonstrated that, in polytopic representations of force feasible sets, most muscle tensions are either fully activated or inactive, and we specified the maximum number of muscles that deviate from this pattern. The second result introduced a novel index to characterize \emph{in silico} torque and force feasible sets, incorporating the geometric processes involved in our set formulation into its calculation.

Chapter \ref{chapter:4} delved into the practical implications of the results established in Chapter \ref{chapter:3}, quantifying the challenges associated with personalizing a musculoskeletal model using force feasible sets represented as ellipsoids or polytopes. Through an \emph{in silico} study, we attempted to personalize various parameters of a musculoskeletal model, assuming a limited number of upper-limb postures and known \emph{in silico} force feasible sets modeled as polytopes or ellipsoids. We introduced a new index, termed the \emph{enlargement complexity}, to evaluate the difficulty of this personalization process. This index was computed through an analysis of the solutions found during an optimization-based personalization process, considering different search spaces. The computed index values suggested that the challenges associated with personalization were not primarily related to the geometric definition of muscle paths, for both force feasible set representations. Thus, this chapter addressed Challenges 2 and 3.

Chapter \ref{chapter:5} directly addressed Challenge 3 by introducing an experimental protocol to gather maximal isometric forces in four different upper-limb postures. The experimental setup was designed to accommodate the anthropometric variability of individuals. This chapter confronted the theoretical ellipsoid representation assumption from Chapter \ref{chapter:3} with \emph{in vivo} data. The central objective was to evaluate whether a 50-muscle upper-limb musculoskeletal model was sufficient to apply theoretical muscle tension interaction assumptions induced by an ellipsoid representation of force feasible sets, which necessitate a large number of muscles to be assumed. Thus, we formulated a hypothesis about the behavior of muscle tension interactions across different postures and conducted an optimization-based personalization process using the participants' experimental measurements. The results indicated that while an ellipsoidal representation of force feasible sets might capture the shape and orientation of \emph{in vivo} maximal isometric force measurements, the theoretical results of Chapter \ref{chapter:3} did not necessarily lead to improved modeling of muscle tension interactions. 

In conclusion, this thesis evaluated the potential of a set-theoretic approach to represent maximal isometric forces at the hand. It examined how this approach addresses experimental challenges and assesses the extent to which it can reveal the biomechanical properties of human upper-limb muscles and their interactions.

% \subsection*{Perspectives}
% \addcontentsline{toc}{section}{Perspectives}

% This section outlines potential avenues for future research that could enhance and extend the findings of this thesis. We first discuss potential improvements to the current framework, followed by a broader discussion of future perspectives and applications.

\subsection*{Perspectives}
\addcontentsline{toc}{section}{Perspectives}

This section outlines potential avenues for future research that could enhance and extend the findings of this thesis. We first discuss potential improvements to the current framework, followed by a broader discussion of future perspectives and applications.

\subsubsection*{Improving the force feasible set framework}
\addcontentsline{toc}{subsection}{Improving the force feasible set framework}

\paragraph*{Extending the scope of biomechanical considerations.} While this thesis explored the knowledge offered by \emph{in silico} and \emph{in vivo} force feasible sets, it was challenged by the set-theoretic nature of these representations. Although primarily theoretical hypotheses were employed, incorporating more specific knowledge of individual biomechanics could improve the adequacy of \emph{in silico} force feasible sets in regard to experimental maximal force measurements. For instance, future work could consider integrating maximal isometric torques alongside maximal isometric forces, as explored in (\cite{rezzougUpperLimbIsometricForce2021b}). Also, knowledge of specific muscle activations related to a force direction could lead to better modeling of tension feasible sets, as their shape across different postures was assumed to be identical in this thesis.

Furthermore, to enhance the realism of force feasible set models, it is essential to consider a broader scope. This thesis focused on characterizing force feasible sets at the hand using an upper-limb musculoskeletal model, but future research could expand this to encompass a wider range of muscle groups within a whole-body model. This would enable the investigation of how different muscle groups contribute to force production and how their interactions influence overall force capabilities. For instance, incorporating trunk and lower limb muscles could provide insights into the role of core stability and lower limb support in generating hand forces.

\paragraph*{From isometric to dynamic force feasible sets.} This thesis focused on isometric maximal force exertions due to their possible experimental measurements. However, human-robot collaboration often involves dynamic movements and postures. While a dynamical description of force feasible sets exists (as discussed in Chapter \ref{chapter:1}), its formulation is more complex, requiring detailed knowledge of muscle properties (such as force-velocity relationships) and individual mass and inertial parameters. Future research should investigate the relationship between isometric force feasible sets (and their ellipsoidal representation) and their dynamic counterparts, assessing the extent to which our theoretical assumptions in a static context generalize to dynamic contexts.

\paragraph*{Exploring moment feasible sets.} This thesis primarily focused on maximal isometric forces, assuming negligible isometric moments. However, moments are inherent to any exertion, even if small. Therefore, generalizing findings such as the ellipsoidal approximation of force feasible sets requires careful consideration. While the theoretical framework developed for maximal isometric forces assumed no moment exertion in theory, this does not relate to experimental measurements. Thus, a geometric analysis is needed to characterize the set formed by the combination of both maximal isometric forces and moments. As distinct mathematical entities, they cannot be simply represented as two ellipsoids within the same space. However, given the geometric relationship between forces and moments (as Plücker coordinates of a screw line (\cite{dorstGeometricAlgebraComputer2007})), a unified geometric 3D characterization of maximal isometric force and moment may be possible.

\paragraph*{Integrating contact forces and stability constraints.}  Explicitly modeling contact forces and stability constraints is crucial for accurately representing the biomechanical constraints on force production, as these factors significantly influence the range and direction of feasible forces while ensuring postural stability during exertion. Maximal force exertions are inherently dependent on interactions with the environment, including contact with the ground and supporting surfaces. As demonstrated in (\cite{leeBiomechanicalAnalysisCoordinated2023}), ground contact forces at the feet can substantially influence hand force production. In our presented experimental setup, where participants were seated to minimize ground contact, they could potentially have utilized their fixing belt to generate greater hand forces. Therefore, future models should explicitly incorporate contact forces, including those involved in the use of a fixing belt, to accurately predict maximal force exertions. This would also allow for the integration of physical constraints related to postural stability, ensuring that the model accurately reflects the biomechanical limitations and compensatory mechanisms involved in maintaining balance during force production. For example, depending on the task and the individual's biomechanics, specific muscle activation patterns might be necessary to maintain stability while exerting maximal force. By considering these additional factors, future force feasible set models can achieve greater realism in characterizing the biomechanical force limits of an individual.

\subsubsection*{Broader perspectives and applications}
\addcontentsline{toc}{subsection}{Broader perspectives and applications}

\paragraph*{Standardizing the measurement of multiple isometric force exertions in a posture.}
Reconstructing an isometric force feasible set relies heavily on the maximal voluntary isometric contraction (MVIC) protocol. We proposed an adaptive setup to accommodate anthropometric variability, but further refinements may be necessary to ensure precise maintenance of upper-limb postures. Our experiments suggest that the cognitive demands of simultaneously maintaining a posture, exerting maximal force, and controlling direction may compromise posture stability. Future studies should investigate how this cognitive overload affects posture stability, force amplitude, and directional accuracy.

\paragraph*{Towards more realistic muscle activation models}
This thesis has explored the geometry and approximation of force feasible sets, providing valuable insights into human force production capabilities. However, our models have relied on simplified representations of muscle activation, neglecting the intricate interplay between muscles and the nervous system. To further enhance the realism and predictive accuracy of force feasible set models, future research could incorporate more nuanced approaches to muscle activation. While this thesis explored idealized models like $\mathcal{T}_\infty$ and $\mathcal{T}_2$ tension set models, representing independent and a specific case of coordinated activations respectively, these fail to capture the full complexity of muscle coordination during force production. Future models could leverage electromyography data to capture the precise timing and intensity of muscle activations in various tasks, potentially using techniques like muscle synergy analysis to identify patterns of muscle co-activation. Additionally, incorporating models of neuromuscular control mechanisms, such as muscle reflexes, proprioceptive feedback, and motor unit recruitment, could provide a more comprehensive representation of human motor control. By integrating these refinements, future models can move beyond simplified representations and capture the intricate dynamics of muscle activation and neuromuscular control, leading to a deeper characterization of human force capacities.

\paragraph*{Applications of force feasible sets in computer-aided design for ergonomics.}
Force feasible sets offer a promising avenue for enhancing computer-aided design (CAD) in ergonomics. By representing the range of forces exertable by a human in various postures, these sets could be integrated into CAD software to provide ergonomic assessments of designs. For instance, when designing a workspace, the software could utilize force feasible sets to evaluate whether the forces required to interact with objects in a workspace are within the user's capacities. This could highlight potential sources of discomfort or strain, enabling designers to modify designs and mitigate the risk of musculoskeletal injuries.

Furthermore, incorporating anthropometric data and task-specific constraints into the generation of customized force feasible sets could lead to more ergonomic and personalized designs. This approach could be particularly valuable in the design of assistive devices or rehabilitation equipment, ensuring that such devices effectively complement the user's capabilities without exceeding their physical limitations. Ultimately, integrating force feasible sets into CAD software could facilitate the development of user-centered designs that promote comfort, efficiency, and safety.

% \subsection*{Perspectives}
% \addcontentsline{toc}{section}{Perspectives}

% This section outlines potential avenues for future research that could enhance and extend the findings of this thesis.

% \paragraph*{Incorporating biomechanical knowledge for improved force feasible set prediction.} 
% While this thesis explored force feasible set prediction through a personalization processes both \emph{in silico} and \emph{in vivo}, these processes were challenged by the set-theoretic nature of force feasible sets. Although primarily theoretical hypotheses were employed, incorporating more specific knowledge of individual biomechanics could improve these personalization processes. For instance, future work could consider integrating maximal isometric torques alongside maximal isometric forces, as explored in \cite{rezzougUpperLimbIsometricForce2021b}. Also, knowledge of specific muscle activations related to a force direction could lead to better modeling of tension feasible sets, as their shape accross different postures was assumed, in this thesis, to be identical.

% \paragraph*{Standardizing the measurement of multiple isometric force exertions in a posture.}
% Reconstructing an isometric force feasible set relies heavily on the maximal voluntary isometric contraction (MVIC) protocol. We proposed an adaptive setup to accommodate anthropometric variability, but further refinements may be necessary to ensure precise maintenance of upper-limb postures. Our experiments suggest that the cognitive demands of simultaneously maintaining a posture, exerting maximal force, and controlling direction may compromise posture stability. Future studies should investigate how this cognitive overload affects posture stability, force amplitude, and directional accuracy.

% \paragraph*{Extending to dynamic force feasible sets.}
% This thesis focused on isometric maximal force exertions due to their relative ease of measurement. However, human-robot collaboration often involves dynamic movements and postures. While a dynamical description of force feasible sets exists (as discussed in Chapter \ref{chapter:1}), its formulation is more complex, requiring detailed knowledge of muscle properties (such as force-velocity relationships) and individual mass and inertial parameters. Future research should investigate the relationship between isometric force feasible sets (and their ellipsoidal representation) and their dynamic counterparts, assessing the extent to which our theoretical assumptions in a static context generalize in dynamic contexts.

% \paragraph*{Exploring moment feasible sets.}
% This thesis primarily focused on maximal isometric forces, assuming negligible isometric moments. However, moments are inherent to any exertion, even if small. Therefore, generalizing findings such as the ellipsoidal approximation of force feasible sets requires careful consideration. While the theoretical framework developed for maximal isometric forces assumed no moment exertion in theory, this does not relate to experimental measurements. Thus, a geometric analysis is needed to characterize the set formed by the combination of both maximal isometric forces and moments. As distinct mathematical entities, they cannot be simply represented as two ellipsoids within the same space. However, given the geometric relationship between forces and moments (as Plücker coordinates of a screw line, \cite{dorstGeometricAlgebraComputer2007}), a unified geometric 3D characterization of maximal isometric force and moment may be possible.

% \paragraph*{Incorporating experimentally observed force mechanisms in force feasible set models.}
% To enhance the realism and predictive accuracy of force feasible set models, it is essential to consider incorporating detailed physiological and biomechanical mechanisms observed in experimental settings. This thesis focused on characterizing force feasible sets at the hand using an upper-limb musculoskeletal model, but future research could expand this scope to encompass a wider range of muscle groups within a whole-body model. This would enable the investigation of how different muscle groups contribute to force production and how their interactions influence overall force capabilities. For instance, incorporating trunk and lower limb muscles could provide insights into the role of core stability and lower limb support in generating hand forces.

% Furthermore, explicitly modeling contact forces and stability constraints is crucial for accurately representing the biomechanical constraints on force production. Maximal force exertions are inherently dependent on interactions with the environment, including contact with the ground, supporting surfaces, and assistive devices. As demonstrated in (\cite{leeBiomechanicalAnalysisCoordinated2023}), ground contact forces at the feet can substantially influence hand force production. Even in our experimental setup, where participants were seated to minimize ground contact, they could still have utilized their fixing belt to generate greater hand forces. Therefore, future models should explicitly incorporate contact forces, including those with assistive devices like the fixing belt, to accurately predict maximal force exertions. This would also allow for the integration of physical constraints related to postural stability, ensuring that the model accurately reflects the biomechanical limitations and compensatory mechanisms involved in maintaining balance during force production. For example, depending on the task and the individual's biomechanics, different muscle activation patterns might be necessary to maintain stability while exerting maximal force. By considering these additional factors, future force feasible set models could achieve greater realism in characterizing the biomechanical force limits of an individual.

% \paragraph*{Towards more realistic muscle activation models}
% This thesis has explored the geometry and approximation of force feasible sets, providing valuable insights into human force production capabilities. However, our models have relied on simplified representations of muscle activation, neglecting the intricate interplay between muscles and the nervous system. To further enhance the realism and predictive accuracy of force feasible set models, future research could incorporate more nuanced approaches to muscle activation. While this thesis explored idealized models like $\mathcal{T}_\infty$ and $\mathcal{T}_2$, representing independent and a specific case of coordinated activations respectively, these fail to capture the full complexity of muscle coordination during force production. Future models could leverage electromyography (EMG) data to capture the precise timing and intensity of muscle activations in various tasks, potentially using techniques like muscle synergy analysis to identify patterns of muscle co-activation. Additionally, incorporating models of neuromuscular control mechanisms, such as muscle reflexes, proprioceptive feedback, and motor unit recruitment, could provide a more comprehensive representation of human motor control. By integrating these refinements, future models can move beyond simplified representations and capture the intricate dynamics of muscle activation and neuromuscular control, leading to a deeper understanding of human movement and its underlying mechanisms.

% \paragraph*{Applications of force feasible sets in computer-aided design for ergonomics.}
% Force feasible sets offer a promising avenue for enhancing computer-aided design (CAD) in ergonomics. By representing the range of forces exertable by a human in various postures, these sets could be integrated into CAD software to provide ergonomic assessments of designs. For instance, when designing a workspace, the software could utilize force feasible sets to evaluate whether the forces required to interact with objects in a workspace are within the user's capacities. This could highlight potential sources of discomfort or strain, enabling designers to modify designs and mitigate the risk of musculoskeletal injuries.

% Furthermore, incorporating anthropometric data and task-specific constraints into the generation of customized force feasible sets could lead to more ergonomic and personalized designs. This approach could be particularly valuable in the design of assistive devices or rehabilitation equipment, ensuring that such devices effectively complement the user's capabilities without exceeding their physical limitations. Ultimately, integrating force feasible sets into CAD software could facilitate the development of  user-centered designs that promote comfort, efficiency, and safety.
