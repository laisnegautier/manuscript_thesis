\pagenumbering{roman}
\setcounter{page}{1}
\pagestyle{plain}
\begin{center}
    \textbf{Remerciements}
\end{center}
Cette thèse est le fruit de \emph{hauts} et de \emph{bas}. Contrairement à toute autre expérience que j'ai pu avoir, le doctorat a la particularité d'offrir une expérience \emph{extrême} de ces hauts et bas.
C'est devant cette opportunité que j'ai décidé non pas de faire la thèse, mais de la vivre. Ces remerciements sont dédiés
à tous ceux qui m'ont aidé à traverser cette expérience.

Les premiers concernés sont mes directeurs de thèse. Jean-Mars Salotti et Nasser Rezzoug, merci
pour votre écoute attentive. Votre confiance sur chacune de mes hypothèses théoriques -- qu'elles soient inutiles, incompréhensibles, ou qui demandèrent plus d'un an de vérification -- m'a 
simplement suffi à atteindre des points culminants de bonheur intellectuel que je n'aurai jamais pu atteindre sans ce cadre que vous m'avez offert.

Je remercie également les membres du jury d'avoir lu et évalué mes travaux de thèse, ainsi que pour les échanges et retours que vous avez fournis avec grand intérêt.

Je remercie l'ensemble de l'équipe Auctus pour m'avoir offert un cadre de travail unique et diversifié. Commençons par David : je ne te l'ai jamais dit mais \emph{non, je ne vois pas ce que tu veux dire.}
Je te remercie de m'avoir convaincu de rester discuter maths après ma journée pendant des heures, alors que j'avais vraiment autre chose de prévu. 
Vincent, quand je rigole lorsque tu fais des contrepétries, en fait je comprends le sens des mots que quelques heures plus tard -- c'est un rire social de réflexe, désolé.
Lucas, je te remercie de m'avoir sorti à de multiples reprises de mon état de torpeur mathématique en me racontant des anecdotes sur ta vie de famille, me rappelant -- de manière nécessaire -- que 
le travail se termine à une heure raisonnable. J'ai apprécié particulièrement ta bonne humeur contagieuse -- même quand tu râlais !
Margot, ta présence et ton extrême bonne humeur (plus que celle de Lucas) font partie de ce qui rend l'équipe vraiment agréable 
à vivre au quotidien (sinon on aurait trop à subir Vincent et ça c'est pas drôle).

Je souhaite aussi remercier mes collègues et amis que je me suis fait durant la thèse. 
\emph{``Je tiens à exprimer toute ma gratitude à Elio, mon collègue et ami, pour son soutien indéfectible tout au long de mon parcours doctoral''}.
De manière plus sérieuse sans ChatGPT, courage pour cette dernière ligne droite ! Ahmed, je pense qu'au final $\det{M} = 0$, pour $M$ une matrice avec un nombre infini de valeurs singulières. Mais c'est sûrement faux car je comprends rien à ton sujet en fait. Fais-toi confiance ! 
Je remercie notre cher Pierre, ou cher rayon de soleil devrais-je dire, car tu as illuminé nos journées par ta présence. Quand tu es parti, il n'y avait que désolation, désespoir, et des articles à écrire.
Merci à Valentine et Clara qui m'ont accompagné
durant la dernière ligne droite expérimentale de la thèse : merci beaucoup pour votre dynamisme et bonne humeur permanente alors que vous étiez surchargées !
Merci à Claire d'être présente dans ma vie de manière aléatoire : à chaque fois que l'on se croise je ne m'y attend pas mais ça signifie que la journée est bonne.
Merci à Loïc d'avoir été souriant et \emph{oklm} avec tout ce qui se passait; ce qui m'a fait relativiser sur le sens de la vie.
Merci Antun, le S\footnote{A prononcer \emph{hess}, avec expiration du \emph{h}.}, le vrai, dont ta bontée naturelle n'a d'égal que toi-même.
Je te remercie pour tes encouragements permanents, même quand ce que je faisais était \emph{vraiment} nul, mais aussi pour l'inspiration que tu m'as apportée pour donner le meilleur de moi-même.
Je remercie aussi Nicolas, la N\footnote{A prononcer \emph{haine}, avec expiration du \emph{h}.}, la vraie, dont nos échanges, qu'ils soient de nature très mathématiques ou très pessimistes m'ont simplement
montré que je n'étais pas seul dans la folie. Maintenant, je voudrai remercier les trois plus gros chiens de la casse que j'ai pu rencontrer dans ma vie. 
Erwannichon, merci d'exister et d'être toujours partant pour domestiquer des autruches. Merci à Alicia également,
et à votre canapé pour m'avoir accueilli après avoir bu trop de jus d'orange. Alexis, merci d'exister et 
de ressembler de plus en plus à un lion (capillairement parlant, niveau charisme on y est pas encore). Merci d'avoir nourri et 
éduqué mes plantes qui seraient mortes depuis longtemps sans toi. Benjamin El Camblor, mon meilleur collègue, merci d'exister même si 
t'as pas passé mon expé. Je suis heureux d'avoir vécu la thèse juste pour le fait d'avoir galéré ensemble.

Merci à mes soeurs, mon frère et mes parents d'avoir compris la difficulté émotionnelle de la thèse et de m'avoir encouragé à la finir.
Merci infiniment à Angèle, qui m'a permis de prendre l'air suffisamment régulièrement pour me ressourcer autrement qu'en lisant des mathématiques. Tu es un vent d'air frais dans ma vie, qui me ressource et dynamise constamment !
Merci à Edouard et toi pour m'avoir accueilli à bras ouverts dans une période compliquée, et pour m'encourager mais aussi me faire prendre conscience que je réussis tout de même des étapes difficiles.
Merci à tous les membres de ma famille qui m'ont soutenu de près comme de loin : Cécile, Fréd, Manuela, Calypso, Régis, Anaïs et mamie Annette. Par contre, toi mamie Marinette, jamais je te remercie.

Depuis mes débuts à l'Université, je me suis fait de fidèles amis qui m'ont aidé à me construire et à m'épanouir dans la vie jusqu'au doctorat.
Merci Suzanne : je ne peux imaginer une vie sans toi et tu es l'une des meilleures choses qui me soit arrivées. 
Merci Ella : tu arrives à me garder dans ta vie mouvementée et j'ai hâte de faire parti de tes aventures très rapidement !
Merci Pauline : tu fais partie de ces personnes qui m'ont grandement marqué et influencé sur plusieurs années, 
tout en ayant pu bien rire et profiter en même temps.
Merci Eulalie, pour notre amité qui s'est forgée avec le temps. Tu as apporté de la gaieté dans ma vie à des moments difficiles, donc je te remercie d'être parmis mes proches amies. 
Merci à Gabriel, dont ta seule présence peux égayer ma journée en un instant. Tu es inspirant à tout point de vue et je ne me lasse pas de me reposer autour d'un café avec toi !
Merci à Fanny, Orlanne et Simon : même en backstage de ma vie pendant la thèse, vous avez su me remonter
le moral à des moments très (trop) difficiles en restant simplement vous-mêmes. J'ai hâte de revenir dans la clique !
Merci à Maurine : tu m'as montré
qu'il y avait tellement d'opportunités à saisir mais que peut importe le choix, on peut toujours être satisfait. Te choisir en temps que coloc était
décidemment une sage décision. Merci à Solène, pour ton soutien constant depuis
que je t'ai rencontrée. On a traversé des situations similaires et ce bout de chemin partagé ne peut que solidifier notre amitié. 
Merci à Martin, pour être toi-même si simplement. 
Merci aux \emph{ombres} : Émile, on a partagé de sacrés moments,
mais surtout 3 ans de vie de coloc ensemble. Les liens qu'on a sont, à mon sens, inébranlables, même si tu pars un jour t'isoler en ermite. 
Benoît, de même, notre amitié reste solide au fil du temps et tu as été le premier à me montrer que je suis entouré de personnes de confiance. 
Merci à Pamplemousse\footnote{\emph{Pamplemousse Henri Benoît}, de son vrai nom.}, ma chatte ingrate qui devrait apprendre à arrêter de gratter sa litière quand je suis en visio.

Pour conclure, merci à mon amour, Sébastien, pour m'avoir écouté parler de polytopes pendant des heures alors qu'on était censé être en vacances. 
Merci de m'avoir demandé 3000 fois \emph{ça avance comme tu veux ?} alors que je te répondais à chaque fois \emph{non}. 
Mais aujourd'hui, je te réponds \emph{oui !}

% PS : tes performances de force sont officiellement supérieures, en général, à celles de Lucas\footnote{Something to brag about.}.

% Avant de passer aux remerciements moins formels, je tiens à souligner qu'intégrer un doctorat peut être significatif d'une montée en classe sociale. Dans mon cas, cette montée a commencé dès le brevet et a progressé linéairement jusqu'à la soutenance de thèse.
% Est-ce qu'il est vraiment pertinent de s'intéresser à ce sujet maintenant ? C'est ma partie remerciements alors je fais ce que je veux. 
% Comprendre l'impact d'une montée sociale extrême est assez méconnu, car il y a peu de témoignages malgré que la mobilité 
% sociale soit théoriquement possible pour tous\footnote{On ne rentrera pas dans le sujet du déterminisme social.}. C'est un sujet qui s'étoffe depuis peu par des analyses de 
% cas\footnote{\emph{cf.} les travaux de la philosophe Chantal Jaquet et/où Laélia Véron et Karine Abiven pour une approche plus linguistique.}. Pour ma part, transiter entre les classes sociales a eu un impact extrèmement 
% nuisible dans ma vie quotidienne, indépendemment de mon entourage et de ma situation. Bien que le syndrome de l'imposteur frappe de toutes ses forces durant la thèse, l'exclusion ressentie de n'importe quelle classe sociale cotoyée est plus pernicieux et me donne ce goût amer du quotidien\footnote{Ce ressenti est très bien détaillé dans \emph{La Place} d'Annie Ernaux (1983).}.
% Je remercie donc tous les lecteurs qui comprendront que mon principal remerciement est dédié à ce qui m'a offert l'opportunité de m'épanouir durant mes sept premières années d'études mais aussi de tout simplement faire des 
% études : le gouvernement français lui-même à travers le système actuel de bourse étudiante et de bourse au mérite; l'Université de Bordeaux, la région Nouvelle-Aquitaine, le programme Erasmus+ et le programme FIdEx pour le système de bourse d'excellence et les offres 
% de mobilité internationale.


\clearpage
\begin{center}
    \textbf{Upper-limb force feasible set:}\\ 
    \textbf{theoretical foundations and musculoskeletal model reconstruction}
\end{center}
\paragraph*{Abstract.} 
% In physical Human-Robot Interaction, where robots and humans can engage in a shared task via a physical point of contact including the robot's end effector and the human's hand, the human's safety needs to be taken into account by the robotical part. This implies that this collaborative system underlyingly assume the knowledge of human's characteristics to provide an adequate robotic assistance.

In physical Human-Robot Interaction, where robots and humans collaborate on shared tasks through physical contact, such as between the robot's end effector and the human's hand, human safety is a primary concern. This necessitates that the collaborative system inherently consider human characteristics to provide appropriate and safe robotic assistance. To achieve this, it is necessary to evaluate the capabilities of the human upper-limb. In biomechanics, these capabilities are defined through force feasible sets at the hand, which represent all the forces a human operator can exert in a given posture. These three-dimensional sets are influenced by individual factors such as anthropometry and muscle strength and the surface of these sets represents the maximum force capabilities of the human upper-limb in all directions. Therefore, force feasible sets are an invaluable tool for guiding robotic assistance, ensuring it respects biomechanical constraints by remaining within the human's exertable force limits.

Force feasible sets are challenging to measure directly but can be partially represented in isometric conditions by measuring maximum exerted forces. Musculoskeletal models, which mathematically represent the human skeleton, joints, and muscles, allow for \emph{in silico} representation of force feasible sets in various postures through geometric operations (Minkowski sum, projection, intersection) on convex sets. However, these operations are computationally expensive. This thesis first focuses on a novel approach to reduce the computational time of one of the most demanding tasks within this framework.

Furthermore, existing \emph{in silico} models often employ various geometric assumptions about how muscle tensions contribute to joint torques, leading to different characterizations of force feasible sets' shapes, including 3D polytopes and ellipsoids. This thesis proposes a unified framework to represent force feasible sets that explicitly incorporates these geometric assumptions. This framework addresses the limitations of current numerical simulations, which struggle to analyze complex scenarios involving more detailed representation of musculoskeletal models and inherently higher computational costs.

In this regard, accurate representation of individual force capabilities requires precise parameterization of musculoskeletal model components. Given the set-theoretic nature of force feasible sets, this thesis introduces an adapted sensitivity analysis tailored to assess the influence of parameters on the geometric properties of force feasible sets. This analysis also highlights the challenges of personalizing musculoskeletal models due to biomechanical inter-variability.

Finally, an experimental protocol was established to confront \emph{in silico} personalization processes with experimentally measured maximal isometric force exertions collected across various postures. Through biomechanical assumptions leading to a computationally less expensive representation of force feasible sets as ellipsoids, muscle parameters personalization is achieved, validating \emph{in vivo} the theoretically-driven results of this thesis.

\textbf{Keywords:} Force feasible set; Musculoskeletal model; Personalization; Force polytope

\clearpage
\begin{center}
    \textbf{Capacités de force du membre supérieur :}\\
    \textbf{fondements théoriques et reconstruction de modèles musculosquelettiques}
\end{center}
\paragraph*{Résumé.} En interaction physique Homme-Robot, un robot et un individu effectuent une tâche de collaboration par le biais d'un contact physique, par exemple entre l'effecteur terminal du robot et la main de l'humain. Une attention nécessaire se porte alors sur la sécurité de cette collaboration. Afin que le robot puisse fournir une assistance adaptée, il est primordial de caractériser les limites d'un individu, notamment celles biomécaniques. Les capacités de force du membre supérieur humain correspondent à l'ensemble des efforts exerçables au niveau de la main, dans une posture spécifique. La forme de ces ensembles reflète des propriétés individuelles, telles que l'anthropométrie et la force musculaire. De plus, leur surface caractérise les efforts maximaux possibles. Par conséquent, la connaissance des capacités de force d'un individu permet d'aiguiller l'assistance du robot afin de respecter les limites en force de l'humain.

Expérimentalement, les capacités de force sont difficiles à mesurer directement. Néanmoins, en conditions isométriques, elles peuvent être décrites partiellement en mesurant des efforts maximaux dans des directions spécifiées. \emph{In silico}, les modèles musculosquelettiques représentent mathématiquement le squelette, les articulations et les muscles. Ils permettent de simuler les capacités de force dans diverses postures, construites par le biais d'opérations géométriques (somme de Minkowski, projection, intersection) sur des ensembles convexes. Cependant, ces opérations sont coûteuses en temps de calcul. Cette thèse se concentre donc en premier lieu sur une nouvelle approche réduisant le temps de calcul de l'une de ces opérations.

Par ailleurs, la modélisation des capacités de force nécessite de faire des hypothèses sur la façon dont les tensions musculaires contribuent aux couples articulaires. Ces hypothèses influencent la forme des capacités de force, notamment leur représentation sous forme de polytope ou d'ellipsoïde. Cette thèse propose donc une unification de ces représentations afin de considérer, de manière géométrique, diverses hypothèses biomécaniques sur les relations entre tensions musculaires. Bien que l'utilisation de modèles musculosquelettiques complexes et détaillés limite la simulation des capacités de force, ce cadre théorique met en évidence une forme universelle pour la représentation des capacités de force d'un individu.

Également, une représentation précise des capacités de force d'un individu nécessite la connaissance d'un modèle musculosquelettique personnalisé. Compte tenu de l'approche ensembliste des capacités de force, cette thèse propose une analyse de sensibilité adaptée évaluant l'influence des paramètres musculaires sur les propriétés géométriques des capacités de force. Cette analyse met également en évidence les défis de la personnalisation en regard de l'inter-variabilité entre individus.

Enfin, un protocole expérimental a été établi afin de confronter les méthodes de personnalisation \emph{in silico} à des forces maximales isométriques mesurées dans différentes postures du membre supérieur. La représentation des capacités de force sous forme d'ellipsoïdes, moins coûteuse en temps de calcul, et l'ajout d'hypothèses biomécaniques sur le comportement musculaire amènent à la personnalisation des muscles d'un modèle musculosquelettique, validant \emph{in vivo} les résultats théoriques proposés dans cette thèse.

\textbf{Mots-clés:} Capacités de force; Modèle musculosquelettique; Personnalisation; Polytope de force
