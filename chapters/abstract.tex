\begin{center}
    \textbf{Upper-limb force feasible set:}\\ 
    \textbf{theoretical foundations and musculoskeletal model reconstruction}
\end{center}
\paragraph*{Abstract.} 
% In physical Human-Robot Interaction, where robots and humans can engage in a shared task via a physical point of contact including the robot's end effector and the human's hand, the human's safety needs to be taken into account by the robotical part. This implies that this collaborative system underlyingly assume the knowledge of human's characteristics to provide an adequate robotic assistance.

In physical Human-Robot Interaction, where robots and humans collaborate on shared tasks through physical contact, such as between the robot's end effector and the human's hand, human safety is a primary concern. This necessitates that the collaborative system inherently consider human characteristics to provide appropriate and safe robotic assistance. To achieve this, it is necessary to evaluate the capabilities of the human upper-limb. In biomechanics, these capabilities are defined through force feasible sets at the hand, which represent all the forces a human operator can exert in a given posture. These three-dimensional sets are influenced by individual factors such as anthropometry and muscle strength and the surface of these sets represents the maximum force capabilities of the human upper-limb in all directions. Therefore, force feasible sets are an invaluable tool for guiding robotic assistance, ensuring it respects biomechanical constraints by remaining within the human's exertable force limits.

Force feasible sets are challenging to measure directly but can be partially represented in isometric conditions by measuring maximum exerted forces. Musculoskeletal models, which mathematically represent the human skeleton, joints, and muscles, allow for in silico representation of force feasible sets in various postures through geometric operations (Minkowski sum, projection, intersection) on convex sets. However, these operations are computationally expensive. This thesis first focuses on a novel approach to reduce the computational time of one of the most demanding tasks within this framework.

Furthermore, existing \emph{in silico} models often employ various geometric assumptions about how muscle tensions contribute to joint torques, leading to different characterizations of force feasible sets' shapes, including 3D polytopes and ellipsoids. This thesis proposes a unified framework to represent force feasible sets that explicitly incorporates these geometric assumptions. This framework addresses the limitations of current numerical simulations, which struggle to analyze complex scenarios involving more detailed representation of musculoskeletal models and inherently higher computational costs.

In this regard, accurate representation of individual force capabilities requires precise parameterization of musculoskeletal model components. Given the set-theoretic nature of force feasible sets, this thesis introduces an adapted sensitivity analysis tailored to assess the influence of parameters on the geometric properties of force feasible sets. This analysis also highlights the challenges of personalizing musculoskeletal models due to biomechanical inter-variability.

Finally, an experimental protocol was established to confront \emph{in silico} personalization processes with experimentally measured maximal isometric force exertions collected across various postures. Through biomechanical assumptions leading to a computationally less expensive representation of force feasible sets as ellipsoids, muscle parameters personalization is achieved, validating \emph{in vivo} the theoretically-driven results of this thesis.

\textbf{Keywords:} Force feasible set; Musculoskeletal model; Personalization; Force polytope

\clearpage
\begin{center}
    \textbf{Capacités de force du membre supérieur :}\\
    \textbf{fondements théoriques et reconstruction de modèles musculosquelettiques}
\end{center}
\paragraph*{Résumé.} En interaction physique Homme-Robot, un robot et un individu effectuent une tâche de collaboration par le biais d'un contact physique, par exemple entre l'effecteur terminal du robot et la main de l'humain. Une attention nécessaire se porte alors sur la sécurité de cette collaboration. Afin que le robot puisse fournir une assistance adaptée, il est primordial de caractériser les limites d'un individu, notamment celles biomécaniques. Les capacités de force du membre supérieur humain correspondent à l'ensemble des efforts exerçables au niveau de la main, dans une posture spécifique. La forme de ces ensembles reflète des propriétés individuelles, telles que l'anthropométrie et la force musculaire. De plus, leur surface caractérise les efforts maximaux possibles. Par conséquent, la connaissance des capacités de force d'un individu permet d'aiguiller l'assistance du robot afin de respecter les limites en force de l'humain.

Expérimentalement, les capacités de force sont difficiles à mesurer directement. Néanmoins, en conditions isométriques, elles peuvent être décrites partiellement en mesurant des efforts maximaux dans des directions spécifiées. \emph{In silico}, les modèles musculosquelettiques représentent mathématiquement le squelette, les articulations et les muscles. Ils permettent de simuler les capacités de force dans diverses postures, construites par le biais d'opérations géométriques (somme de Minkowski, projection, intersection) sur des ensembles convexes. Cependant, ces opérations sont coûteuses en temps de calcul. Cette thèse se concentre donc en premier lieu sur une nouvelle approche réduisant le temps de calcul de l'une de ces opérations.

Par ailleurs, la modélisation des capacités de force nécessite de faire des hypothèses sur la façon dont les tensions musculaires contribuent aux couples articulaires. Ces hypothèses influencent la forme des capacités de force, notamment leur représentation sous forme de polytope ou d'ellipsoïde. Cette thèse propose donc une unification de ces représentations afin de considérer, de manière géométrique, diverses hypothèses biomécaniques sur les relations entre tensions musculaires. Bien que l'utilisation de modèles musculosquelettiques complexes et détaillés limite la simulation des capacités de force, ce cadre théorique met en évidence une forme universelle pour la représentation des capacités de force d'un individu.

Également, une représentation précise des capacités de force d'un individu nécessite la connaissance d'un modèle musculosquelettique personnalisé. Compte tenu de l'approche ensembliste des capacités de force, cette thèse propose une analyse de sensibilité adaptée évaluant l'influence des paramètres musculaires sur les propriétés géométriques des capacités de force. Cette analyse met également en évidence les défis de la personnalisation en regard de l'inter-variabilité entre individus.

Enfin, un protocole expérimental a été établi afin de confronter les méthodes de personnalisation \emph{in silico} à des forces maximales isométriques mesurées dans différentes postures du membre supérieur. La représentation des capacités de force sous forme d'ellipsoïdes, moins coûteuse en temps de calcul, et l'ajout d'hypothèses biomécaniques sur le comportement musculaire amènent à la personnalisation des muscles d'un modèle musculosquelettique, validant \emph{in vivo} les résultats théoriques proposés dans cette thèse.

\textbf{Mots-clés:} Capacités de force; Modèle musculosquelettique; Personnalisation; Polytope de force